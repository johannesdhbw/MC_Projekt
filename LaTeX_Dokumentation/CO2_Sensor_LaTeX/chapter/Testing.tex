\label{Testing}

\begin{table}[!hbt]
	
	\centering
	
	\begin{tabular}{|p{5cm}|p{5cm}|}
		
		\hline
		Projekt: CO2-Sensor & Datum: \\
		\hline
		ID: CO201 & Version: 1.0 \\
		\hline
		\multicolumn{2}{|l|}{Titel: Visualisierung der Luftqualität auf Basis von CO2-Grenzen} \\
		\hline
		Items: void ask(int) & TestKfg: 01 \\
		\hline
		\multicolumn{2}{|p{\textwidth-2\tabcolsep}|}{Zielsetzung: Der Test soll zeigen, dass die Software die gemessenen CO2-Werte richtig interpretieren kann.} \\
		\hline
		\multicolumn{2}{|l|}{Anforderungen: R01} \\
		\hline
		\multicolumn{2}{|l|}{Erforderliche Inputs zu Testbeginn: CO2 Werte} \\
		\hline
			
	\end{tabular}

\captionabove{Test 1}
\label{tab:Test_1}

\end{table}

\begin{table}[!hbt]
	
	\centering
	
	\begin{tabular}{|p{7.4cm}|p{7.4cm}|}
	
		\hline
		Tester: & Beobachter: \\
		\hline
		\multicolumn{2}{|l|}{Protokolldatei: } \\
		\hline
		Status: & Problembericht: \\
		\hline
	
	\end{tabular}

	\captionabove{Tester 1}
	\label{tab:Tester1}

\end{table}


\begin{table}[!hbt]
	
	\centering
	
	\begin{tabular}{|p{5cm}|p{5cm}|}
		
		\hline
		Projekt: CO2-Sensor & Datum: \\
		\hline
		ID: CO202 & Version: 1.0 \\
		\hline
		\multicolumn{2}{|l|}{Titel: Auswahl von verschiedenen Messprofilen} \\
		\hline
		Items: void ask(int) & TestKfg: 01 \\
		\hline
		\multicolumn{2}{|p{\textwidth-2\tabcolsep}|}{Zielsetzung: Der Test soll zeigen, dass der Anwender zwischen drei verschiedenen Messprofilen wählen kann.} \\
		\hline
		\multicolumn{2}{|l|}{Anforderungen: R03} \\
		\hline
		\multicolumn{2}{|l|}{Erforderliche Inputs zu Testbeginn: Anwender} \\
		\hline
		
	\end{tabular}
	
	\captionabove{Test 2}
	\label{tab:Test_2}
	
\end{table}

\begin{table}[!hbt]
	
	\centering
	
	\begin{tabular}{|p{8cm}|p{8cm}|}
		
		\hline
		Tester: Alexander Herrmann & Beobachter: Johannes Ruffer \\
		\hline
		\multicolumn{2}{|p{\textwidth-2\tabcolsep}|}{Protokoll: \newline Zunächst wurde der Arduino an den Laptop angeschlossen und somit das aktuelle Programm gestartet. Nach Auswahl des Schreibemodus, wurde das Messprofilmenü angezeigt. Nach drei UP-Button-Klicks, wurde der Messmodus 1 (Echtzeitmodus) mit dem Enter-Button bestätigt. \newline Die anschließende Messung wurde in Echtzeit auf dem LCD-Display ausgegeben. Die abgespeicherten Messungen wurden ebenfalls in Echtzeit gemessen. Anschließend wurde das Programm neu gestartet, um den Modus 2 auszuwählen. Auch dieser wurde erfolgreich mit den richtigen Zeitintervallen durchgeführt. \newline Der dritte Start des Programms diente dazu, die Tagesmessung auszuprobieren. Auch hier hat die Auswahl und Durchführung in Bezug auf die Testanforderungen erfolgreich funktioniert.} \\
		\hline
		Status: Erfolgreich & Problembericht: Nicht vorhanden \\
		\hline
		
	\end{tabular}
	
	\captionabove{Tester 2}
	\label{tab:Tester2}
	
\end{table}
