\label{Softwaremplementierung}



Für die entgültige Struktur der Software waren mehrere Anforderungen ausschlaggebend. \\
Zu Beginn des Programmablaufs wird der Anwender gefragt, ob er eine neue Messung starten, oder lieber die letzte auslesen möchte. Ist das Auslesen der letzten Messung gewünscht, erleuchten die LEDs grün, gelb und rot. Dies soll dem Anwender die Bestätigung geben, dass die letzte Messung ausgelesen werden kann. Softwaretechnisch passiert an dieser Stelle nichts weiter. Diese Auswahl hat den alleinigen Zweck, dass der Benutzer die Mikro-SD-Karte nicht während einer Messung herausnimmt und somit einen Fehler provoziert. \\
Falls der Anwender dennoch gegen die Vorgaben der Entwickler handeln sollte und die Mikro-SD-Karte während einer Messung herausnimmt, wird diese trotzdem bis zum Ende durchgeführt. Die resultierende Konsequenz daraus ist, dass sich auf der Mikro-SD-Karte keine Datei befindet. \\
Handelt der Anwender im Sinne der Entwickler und wählt zum Entfernen der Mikro-SD-Karte den Lese-Modus, kann er nachdem diese wieder im Adapter ihren Platz gefunden hat, den Vorgang mit dem Enter-Button bestätigen. So öffnet sich wieder das Hauptmenü und eine neue Messung kann auf Wunsch gestartet werden. \\
Als nächstes wurden drei verschiedene Messprofile definiert und implementiert, sodass der Benutzer zwischen einer Echtzeit-, Stunden- und Tagesmessung wählen kann. In diesen Messprofilen ist definiert, in welchen Abständen und wie lange Messungen durchgeführt werden sollen. Somit konnten Anforderung Nummer 1 und 10 aus der Tabelle \ref{tab:Anforderungen} gemeinsam umgesetzt werden. \\
Nach der Wahl des Messprofils muss der Arduino den CO\textsubscript{2}-Sensor ansteuern und richtig konfigurieren. Es muss getestet werden, ob er funktionstüchtig und bereit ist, eine Messung zu starten. Zudem kann der verwendete Sensor nicht nur CO\textsubscript{2}-Werte, sondern beispielsweise auch Temperaturen messen, sodass der Arduino die richtigen Werte anfordern muss. Damit dies möglich ist, mussten wir die Bibliothek <Adafruit\_CCS811.h> einbinden. Genaueres zu diesem Algorithmus wird in Kapitel \ref{CCS811} erläutert. \\
Zunächst wird im Setup geprüft, ob der Sensor gestartet werden kann. Falls dies nicht der Fall sein sollte, wird eine Fehlermeldung ausgegeben. Nach erfolgreichem Start ist der Arduino angehalten so lange mit dem Programm zu warten, bis der Sensor zurückmeldet, dass er bereit ist, die Messung zu beginnen. \\
Auch während dem Programmdurchlauf wird bei jeder neuen Messung kontrolliert, ob der CO\textsubscript{2}-Sensor funktionstüchtig ist. Danach wird mithilfe der oben genannten Bibliothek der CO\textsubscript{2}-Wert gemessen und an den Arduino weitergegeben. \\
Nach Einlesen der Daten, vergleicht der Mikrocomputer diese mit den gegebenen Grenzwerten. Je nach Bewertung des gemessenen Wertes wird eine der grün/gelb/roten \ac{LED}s eingeschaltet. Auch die blaue \ac{LED}, welche die Ansteuerung des automatisierten Fenstersscheibenmotors simulieren soll, wird je nach Messwert an- oder ausgeschaltet. \\
Zudem werden dem Anwender die jeweiligen Daten im \ac{LCD} Display aufgegeben, was durch die Bibliothek <LiquidCrystal.h> möglich ist. \\
Damit der Verlauf der Messung später auf Excel geplottet werden kann, wird auf einer Mikro-SD-Karte der gemessene Wert im .csv-Format als .txt-Datei abgespeichert. Das bedeutet, dass die Messungen in einer Zeile, getrennt durch Kommas, gesichert werden. Falls Excel trotzdem alle Werte in ein Feld schreiben sollte, anstatt jeden Wert in einem solchen Feld anzuzeigen, kann dieses Problem innerhalb von Excel berichtigt werden. \\
In der Menüleiste von Excel muss das Feld <Daten> ausgewählt werden. Durch die Markierung des Eintrags und Auswählen des Feldes <Text in Spalten>, können die Trennzeichen händisch definiert werden. Nachdem Kommas als Trennzeichen ausgewählt wurden, teilen sich die Messungen auf Spalten auf. Nun kann, durch Makierung aller Messwerte und der Auswahl des gewünschten Diagramms, der Verlauf der Messung dargestellt werden. \\
Nachdem der letzte Wert auf der Mikro-SD-Karte gespeichert wurde, beginnt das Programm durch die Anzeige des Hauptmenüs, automatisch von vorne. Falls der Sleep-Mode-Button während des Programmablaufs gedrückt wurde, springt das Programm in einen Art Stromspar- oder auch Schlafmodus. Hier gehen sowohl alle Lampen, als auch die Beleuchtung des \ac{LCD}-Displayswq aus. Das Wieder-Aufwecken der Hardware geschieht durch Betätigung des Enter-Buttons. Das Bestätigen eines anderen Tasters wird ignoriert. Nach Verlassen des Sleep-Modes, erscheint wieder das Hauptmenü. Das Programm kann nun erneut von vorne gestartet werden. \\