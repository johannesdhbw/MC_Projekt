\label{FazitSW}

Nachdem wir uns auf ein Projekt geeinigt hatten, wurden alle Anforderungen dafür definiert. Die Planung konnte anschließend erfolgen, in welcher wir uns überlegt haben, welche Komponenten wir für die Umsetzung benötigen. \\
Während der Softwareentwicklung ergab sich dann trotz dem anfangs entworfenem Schaltungslayout, dass wir zu wenig digitale PINs am Arduino zu Verfügung hatten. Diese späte Erkenntnis kam zu Stande, da wir während der Entwicklung festgestellt haben, dass an die Arduino-PINs Null und Eins keine Hardwarekomponenten angeschlossen werden können. Auch an den PIN AREF, konnten keine von uns verwendeten Komponenten angeschlossen werden. Somit fehlten Steckmöglichkeiten, sodass wir kurzfristig einen I$_2$C-Adapter an das \ac{LCD}-Display anschließen mussten. So konnten wir zum einen Anschlussmöglichkeiten am Arduino sparen, zum anderen wird der Adapter mit analogen PINs verbunden, sodass uns umso mehr digitale PINs zu Verfügung standen. \\
Eine weitere Änderung im Layout musste aufgrund der Interrupt-Funktion des Sleep-Mode-Buttons vorgenommen werden. Die Tatsache, dass nur die digitalten PINs Zwei und Drei am Arduino für Interrupts genutzt werden können, war uns anfangs nicht bewusst. Somit wurde auch hier während der Entwicklung das in der Planung gezeichnete Schaltungslayout geändert. \\
Die softwaretechnische Integration des CO$_2$-Sensors, hat sich aufgrund der Suche nach einer geeigneten Bibliothek und dem anschließenden Einarbeiten in jene, zeitlich etwas gestreckt. \\
Auch in der softwaretechnischen Integration des Mikro-SD-Karten-Adapters, haben sich der zeitliche Aufwand von Finden und Verstehen einer passenden Bibliothek, bemerkbar gemacht. \\
Abschließend kann man sagen, dass trotz kleinen Komplikationen während der Softwareentwicklung, das Projekt dank einer guten Planung erfolgreich umgesetzt wurde. 