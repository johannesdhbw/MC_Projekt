\label{Anforderungen}

Damit die Bewertung des Projektes erfolgreich wird, müssen zu Projektbeginn Anforderungen erarbeitet und festgelegt werden. Diese sind unveränderbar, da das Ergebnis sonst verfälschen würde.

\begin{table}[!hbt]
	
	\centering
	
	\begin{tabular}{|r| p{8.4cm}|p{4.7cm}|}
		
		\hline
		Nummer & Anforderungen & Verifikationsmethode \\
		\hline
		1 & Echtzeitmessung der Luftgüte & Measurement \\
		\hline
		2 & Mindestmessbereich von 400 ppm bis 5000 ppm & Review \\
		\hline
		3 & Visualisierung der Luftgüte mithilfe von LEDs (gut, mittel, schlecht) & Test \\
		\hline
		4 & Ausgabe der Luftgüte mithilfe von \ac{LCD}-Display & Test \\
		\hline
		5 & Ansteuern eines Fensterscheibenmotors mithilfe einer LED simulieren & Test \\
		\hline
		6 & Bei schlechter Luftgüte: Fenster öffnet sich (LED an) & Test \\
		\hline
		7 & Bei guter Luftgüte: Fenster schließt sich (LED aus) & Test \\
		\hline
		8 & Speichern im CSV-Format & Test, Analysis\\
		\hline
		9 & Zugriff auf Messdaten über SD-Karte & Test \\
		\hline
		10 & Benutzer kann zwischen drei Messprofilen auswählen (Messprofil: Abtastrate) & Test \\
		\hline
		
	\end{tabular}

\captionabove{Anforderungen an das Projekt}
\label{tab:Anforderungen}

\end{table}

Eine Anforderung muss messbar und/oder überprüfbar sein, um deren Umsetzung später objektiv bewerten zu können. Somit wird auch eine sogenannte Verifikationsmethode festgelegt, mit der die Anforderungen später überprüft werden kann. \\
Es gibt sechs verschiedene Verifikationsmethoden:

\begin{itemize}
	\item Similarity: Suche und Abgleich mit bereits vorhandenen Lösungen \cite[S. 122]{HelgaMeyer.}
	\item Insparity: Soll- und Ist-Abgleich von Ein- und Ausgängen nach einem formalen Ablauf (Ein- und Ausgangskriterien sind vorher definiert) \cite[vgl. S. 308]{PeterLiggesmeyer.2009}
	\item Review: Soll- und Ist-Abgleich von Ein- und Ausgängen ohne formalen Ablauf (Ein- und Ausgangskriterien sind vorher nicht definiert) \cite[vgl. S. 317]{PeterLiggesmeyer.2009}
	\item Measurement: Durchführung einer Reihe von Operationen, die das Ziel haben, einen Wert einer quantitativen oder kategorialen Darstellung eines oder mehrerer Attribute zu bestimmen \cite[vgl. S. 395]{DepartmentofResearch&DevelopmentDepartmentofInformationTechnologiesandSystems.}
	\item Analysis: Basiert auf Heuristiken und Statistiken [...] [, die man] sich als starke Compiler-Typisierung [...] im Rahmen einer ausführlichen Datenfluss-Analyse vorstellen [kann] \cite[vgl. S. 4]{JayAbrahamPaulJonesRaoulJetley.}
	\item Test: Prüft und bewertet Software auf Erfüllung der für ihren Einsatz definierten Anforderungen und misst ihre Qualität \cite{Wikipedia.01.03.2020}
\end{itemize}
