\label{Anforderungen}

Damit die Bewertung des Projektes erfolgreich wird, müssen zu Projektbeginn Anforderungen erarbeitet und festgelegt werden. Diese sind unveränderbar, da das Ergebnis sonst verfälschen würde. \\

\begin{table}[!hbt]
	
	\centering
	
	\begin{tabular}{|r| p{8.4cm}|p{4.7cm}|}
		
		\hline
		Nummer & Anforderungen & Verifikationsmethode \\
		\hline
		1 & Echtzeitmessung der Luftgüte & Measurement \\
		\hline
		2 & Mindestmessbereich von 300 ppm bis 3000 ppm & Review \\
		\hline
		3 & Visualisierung der Luftgüte mithilfe von LEDs (gut, mittel, schlecht) & Test \\
		\hline
		4 & Ausgabe der Luftgüte mithilfe von LCD-Display & Test \\
		\hline
		5 & Ansteuern eines Fensterscheibenmotors mithilfe einer LED simulieren & Test \\
		\hline
		6 & Bei schlechter Luftgüte: Fenster öffnet sich (LED an) & Test \\
		\hline
		7 & Bei guter Luftgüte: Fenster schließt sich (LED aus) & Test \\
		\hline
		8 & Speichern im CSV-Format & Test, Analysis\\
		\hline
		9 & Externe Abfrage über USB-Schnittstelle & Test \\
		\hline
		10 & Benutzer kann zwischen drei Messprofilen auswählen (Messprofil: Abtastrate) & Test \\
		\hline
		
	\end{tabular}

\captionabove{Anforderungen an das Projekt}
\label{tab:Anforderungen}

\end{table}