\label{AusblickSW}

Hinsichtlich der Softwareentwicklung wäre es möglich, das \ac{LCD}-Display durch ein \ac{OLED}-Display zu ersetzen. Diese Änderung kann zu einer anschaulicheren Darstellung für den Benutzer eingesetzt werden. Ein Beispiel hierfür wäre eine übersichtlicher Darstellung der Menüanzeige. \\
Auch das Speichern der, zu den Messungen passenden Uhrzeiten, wäre eine Ergänzung zu dem jetzigen Format. So könnte das plotten der Graphiken nicht über die Anzahl der Messungen, sondern über die Zeit ermöglicht werden. Dazu müsste jedoch garantiert sein, dass der Arduino über eine Internet- oder Computerverbindung verfügt, sodass er die korrekte Uhrzeit auslesen kann. \\
Zudem könnte der Arduino das Plotten der Messungen in Form von Graphiken übernehmen. Diese könnten beispielsweise auf der bereits implementierten Mikro-SD-Karte abgespeichert werden. \\
Zu guter Letzt könnte die zeitliche Ansteuerung des Fensterscheibenmotors optimiert werden. Da der Fensterscheibenmotor bisher mit der roten \ac{LED} gleichgeschaltet ist, könnte es zu einem Toggeln zwischen mäßgier und schlechter Luftqualität kommen. Dies würde durch das Trennen dieser Gleichschaltung verhindert werden. Zudem würde die Räumlichkeit, bis zum Erreichen einer guten Luftqualität, gelüftet werden, falls dies gewünscht ist. Dazu müsste die Funktion, welche für die Interpretation der Luftgüte und somit für das An- und Ausschalten der \ac{LED}s zuständig ist, angepasst werden. Die Funktion muss dafür vom Grundaufbau geändert werden. Dies hat zur Folge, dass die Funktion sofort komplexer wird. In diesem Zusammenhang fände das Auslagern und Aufrufen jener Funktion nicht mehr in dieser Form statt.