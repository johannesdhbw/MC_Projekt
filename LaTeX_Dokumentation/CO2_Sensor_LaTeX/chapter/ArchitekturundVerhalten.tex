\label{ArchitekturundVerhalten}

Im folgenden werden drei Diagramme, welche wir im Laufe der Projektplanung anfertigten, beschrieben und erklärt. Diese sind die Grundsteine unseres Projektentwurfs. \\
Das Zustandsdiagramm ist dazu da, die Zustände, in welches sich das Objekt befindet, zu präsentieren. Auch die Übergänge zwischen den einzelnen Zuständen wird hier deutlich. Zudem zeigt ein Zustandsdiagramm den Anfangs- und Endpunkt einer Sequenz von Zustandsänderungen. \cite[vgl. S. 120]{JosephSchmuller.2000} \\
Die Top-Level-Architektur soll anschließend alle Ein- und Ausgänge der Hardware aufzeigen. \\
Zu guter Letzt zeigt das Komponentendiagramm, wie der Name schon verrät, alle Komponenten, welche in Bezug auf das Projekt benötigt werden an. Hier sind sowohl Hard-, als auch Software-Komponenten mit den jeweiligen Verbindungen integriert.