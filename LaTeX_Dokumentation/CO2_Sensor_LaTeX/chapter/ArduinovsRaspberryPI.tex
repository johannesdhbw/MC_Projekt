\label{arduinovsRasbPI}

Nach dem Definieren der Anforderungen musste entschieden werden, mit welchem Mikrocontroller oder auch Einplatinencomputer das Projekt umgesetzt werden soll. Hierbei wurden ein Arduino und ein Raspberry PI in Betracht gezogen. \\
Die Vorteile des Arduinos liegen darin, dass ein sofort einsatzbereites Hardware-/Software-Setup zu Verfügung steht. Des Weiteren hat diser Mikrocontroller eine eigene Entwicklungsumgebung mit plattformübergreifenden Bibliotheken, von welchen wir Gebrauch machen müssen. Zudem ist Arduino eine Plattform, auf Basis von Open-Source-Lizenzen. Dies hat den Vorteil, dass jeder Entwickler sowohl die Quellcodes der Software, als auch die Pläne der Hardware einsehen und für das jeweilige Projekt individuell anpassen kann. \cite[Vgl.]{MircoLang.2018} Nachteile, wie die kostspielige Aufrüstung von Shields sind für unsere Verwendungen nicht von Bedeutung. \\
Vielmehr überwiegten die Nachteile des Raspberry PI, für welchen man kostenpflichtige Zusatzteile für den eigenständigen Betrieb benötigt. Auch die Vorteile des Einplatinencomputers, wie die Netzwerkfähigkeit und den standardmäßigen \ac{HDMI}-Anschluss, werden in diesem Projekt nicht benötigt und spielen für uns somit keine Rolle. \cite[Vgl.]{IONOS.2018} \\
All diese Punkte führten letztendlich zu der Entscheidung, einen Arduino und keinen Rapberry PI für das Projekt zu nehmen. \\
