% Kurzfassung - Autor:

\label{kurzfassung}

Der Bericht wurde von drei Studierenden an der \ac{DHBW}-Ravensburg Campus Friedrichshafen im Rahmen der Mikrocomputertechnik Vorlesung eigenständig von der Projektplanung, über die Durchführung, bis hin zum Projektabschluss mit Dokumentation durchgeführt. \\
Ziel der Arbeit ist es, durch die Praxiserfahrung mit Mikrocomputern, Fähigkeiten und Wissen in diesem Bereich zu erwerben. Durch das Vergleichen von anfänglichen Kosten- und Komplexitätsschätzungen mit den späteren Ergebnissen in der Umsetzung können die Studierenden ein Fazit ziehen, inwiefern diese übereinstimmen. So werden auch Fähigkeiten im Bereich des Projektmanagements weiterentwickelt. \\
Dieses Projekt befasst sich mit der Simulation eines Fensterhebers. Dabei wird das Öffnen und Schließen der Fenster mithilfe einer \ac{LED} simuliert, welche nach der Überschreitung eines bestimmten Grenzwertes aufleuchtet. Sobald nach ausreichendem Lüften die Luftgüte unter einen Schwellwert gerät, soll die \ac{LED} wieder aus gehen. \\
Damit dies möglich ist, wurde die Entwicklung von Software-Code, sowie ein Schaltungslayout und der 3D-Druck des Gehäuses selbständig vorgenommen. \\
Der folgende Bericht dokumentiert die Motivation, Vorgehensweise und Umsetzung von der Projektplanung, über die Implementierung von Hardware und Software, bis hin zu durchgeführten Tests und dem Projektabschluss.