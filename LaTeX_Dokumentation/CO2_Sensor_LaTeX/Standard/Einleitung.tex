% Autor:

\label{Einleitung}

Mithilfe eines Arduinos wurde in diesem Projekt ein CO2-gesteuerter Fensterheber simuliert, welcher dazu dienen soll die Räumlichkeiten bei schlechter Luftqualität automatisch zu lüften. Auch das automatische schließen des Fensters nach Wiederherstellung von guter Luftgüte wird simuliert. \\
Das Öffnen und Schließen der Fenster wird anhand von einer LED simuliert, welche nach der Überschreitung eines bestimmten Grenzwertes aufleuchtet. Sobald nach ausreichendem Lüften die Luftgüte unter einen Schwellwert gerät, soll die LED wieder aus gehen. \\
Damit dies möglich ist, wurde die Entwicklung von Software-Code, sowie ein Schaltungslayout und der 3D-Druck des Gehäuses selbständig vorgenommen. \\
Der folgende Bericht dokumentiert die Vorgehensweisen und Umsetzung von der Projektplanung, über die Implementierung von Hardware und Software, bis hin zu durchgeführten Tests und den Projektabschluss.