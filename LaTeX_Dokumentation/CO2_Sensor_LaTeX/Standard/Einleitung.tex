% Autor:

\label{Einleitung}
Kohlenstoffdioxid ist ein Spurengas, welches im Durchschnitt 0,04'Prozent' der Luftzusammensetzung einnimmt, trotz dieses nur kleinen prozentualen Anteils wird es aufgrund der Bedeutung für unser Ökosystem zu den Hauptbestandteilen der Luft gezählt. Die übliche Angabe des Gehalts in der Luft erfolgt in ppm, so liegt der Durchschnittswert etwa bei 400ppm. Bis zu einem Wert von 800ppm spricht man von einer guten Luftqualität, von 1000-1400ppm gilt die Qualität als mäßig.\\
Erste Einwirkungen auf den Menschen sind um 1300ppm zu vermerken, hier noch in erster Linie durch Konzentrationsschwächen und Schläfrigkeit. Gesundheitlich bedenklich wird es ab ca. 5500ppm. Auswirkungen auf die Atmung entstehen bei einer Konzentration von 30000ppm, was dem Kohlenstoffdioxidgehalt in einem Atemzug entspricht.\\
Im Alltag werden solch hohe Werte normalerweise jedoch nicht erreicht, solange in regelmäßigen Abständen gelüftet wird und man sich nicht unnötig in engen überfüllten Räumlichkeiten aufhält. In Büros zum Beispiel, in denen sich viele Menschen befinden ist ein Wert zwischen 5000 und 6000ppm nicht unüblich.\\
Stellt man sich nun ein Klassenzimmer voller Studierender vor, die den ganzen Tag ihren Vorlesungen folgen, ist schnell ersichtlich, dass ohne ausreichendes Lüften keine geeigneten Lehrbedingungen geschaffen werden können. Um dieses Problem anzugehen wurde in diesem Projekt ein CO2-gesteuerter Fensterheber, mit Hilfe eines Arduinos, simuliert. Durch diesen soll es möglich sein optimale Bedingungen im Sinne der Luftqualität bieten zu können. \\